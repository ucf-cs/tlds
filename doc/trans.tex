\documentclass[10pt,journal,letter,compsoc]{IEEEtran}
\usepackage{booktabs}
\usepackage{tabularx}
\usepackage[english]{babel}
\usepackage[ruled]{algorithm}
\usepackage{algpseudocode}
\usepackage{color}
\usepackage{enumitem}
\let\proof\relax
\let\endproof\relax 
\usepackage{amsthm}
\usepackage[cmex10]{amsmath}
\usepackage{amsfonts}
\usepackage{multicol}
\let\subcaption\relax
\usepackage[font=small]{subcaption}
\usepackage{graphicx}
\usepackage[hidelinks,bookmarks=false,pdfpagelabels]{hyperref}
\usepackage[nocompress]{cite}

% definition environment
\newtheorem{definition}{Definition}
\newtheorem{theorem}{Theorem}
\newtheorem{lemma}{Lemma}
\newtheorem{invariant}{Invariant}

% typewriter font family that support hyphenation
%\newcommand\textvtt[1]{{\normalfont\fontfamily{cmvtt}\selectfont #1}}
\hyphenation{MD-List} 
\hyphenation{skip-list}

\algtext*{EndWhile}
\algtext*{EndFor}
\algtext*{EndIf}
\algtext*{EndFunction}
\newcommand\NIL{\text{NIL}}
\newcommand\TRUE{\text{\textbf{true}}}
\newcommand\FALSE{\text{\textbf{false}}}
\newcommand\BREAK{\text{\textbf{break}}}
\newcommand\CONTINUE{\text{\textbf{continue}}}
\newcommand\AND{\;\text{\textbf{and}}\;}
\newcommand\OR{\;\text{\textbf{or}}\;}
\algrenewcommand\algorithmicindent{1em}

\algblockdefx[StructBlock]{Struct}{EndStruct} [1]{\textbf{struct} #1} [0]{}
\algtext*{EndStruct}
\algblockdefx[ClassBlock]{Class}{EndClass} [1]{\textbf{class} #1} [0]{}
\algtext*{EndClass}
\algblockdefx[MacroBlock]{Define}{EndDefine} [2]{\textbf{define} #1(#2)} [0]{}
\algtext*{EndDefine}
\algblockdefx[InlineBlock]{Inline}{EndInline} [2]{\textbf{inline function} \textsc{#1}(#2)} [0]{}
\algtext*{EndInline}

\begin{document}

\title{Lock-free Transactions for Node-based Concurrent Sets}

\author{Deli~Zhang and~Damian~Dechev%
\IEEEcompsocitemizethanks{\IEEEcompsocthanksitem The authors are with the Department of Electrical Engineering and Computer Science, University of Central Florida, Orlando, FL, 32826. \protect\\
E-mail: de-li.zhang@knights.ucf.edu and dechev@eecs.ucf.edu}}

\IEEEtitleabstractindextext{
\begin{abstract}
    asd
\end{abstract}

%\category{D.1.3}{Concurrent Programming}{Algorithms}
%\terms{Algorithms, Performance}
\begin{IEEEkeywords}
    Concurrent Data Structure, Transactional Memory, Lock-free 
\end{IEEEkeywords}
}
\maketitle

\section{Introduction}
Operation in lock-free data structures are not composable

transactional set are important for in-memory databases

Generic solution exist for constructing transactional objects, but its either blocking or present large overhead

We present a method to design lock-free transactional objects based-on linked nodes.
We leverage the special knowledge of this type of data structures to apply optimistic synchronization techniques that improves performance. 


\section{Related Work}
\label{sec:related}

General constructions
STM with large overhead, can be lock-free (RSTM), 

HTM small overhead, not suitable for high contention, or large transactions

OTB, black box design, forfiet data structure specific optimization, need to reverse operation upon failure

Custom constructions
PTB, white box, still use semantic read/write set. Need to reverse operation upon failure

Our approach, do not track read/write set, highly concurrent, Optimistic lock-free

Concurrency control
    Two-phase locking
    MRLock

lowering the overhead of non-blocking transactional memory
\section{Conclusion}
\label{sec:conclusion}
\bibliographystyle{abbrv}
\bibliography{citation}

\end{document}
