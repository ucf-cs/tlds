In this section, the PI expalins the three key ideas of the proposed approach by an example: 1) node-based semantic conflict detection; 2) interpretation-based logical rollback; and 3) cooperative transaction execution.
For clarity, the constants and data type definitions are listed in Algorithm~\ref{alg:nodestructure}.
In addition to the fields used by the base lock-free data structure, a new field $info$ is introduced in \textsc{Node}.
\textsc{NodeInfo} stores $desc$, a reference to the shared transaction descriptor, and an index $opid$, which provides a history record on the last access (i.e., given a node $n$, $n.info.desc.ops[n.desc.opid]$ is the most recent operation that accessed it).
A node is considered \emph{active} when the last transaction that accessed the node had an active status (i.e., $n.info.desc.status=\text{Active}$).
A descriptor~\cite{herlihy2012art} is a shared object commonly used in lock-free programming to announce steps that cannot be done by a single atomic primitive.
The functionalities of the transaction descriptor is twofold: a) it stores all the necessary context for helping finish a delayed transaction; and b) it shares the transaction status among all nodes participating in the same transaction.

\vspace{-0.1in}
\begin{algorithm}[h]
    \caption{Type Definitions}
    \label{alg:nodestructure}
    \vspace{-0.2in}
    \begin{multicols}{2}
        \begin{algorithmic}[1]
            \Enum{TxStatus}
            \State Active
            \State Commited
            \State Aborted
            \EndEnum
            \Enum{OpType}
            \State Insert
            \State Delete
            \State Find
            \EndEnum
            \Struct{Operation}
            \State \textbf{OpType} $type$
            \State \textbf{int} $key$
            \EndStruct
            \Struct{Desc}
            \State \textbf{int} $size$
            \State \textbf{TxStauts} $status$
            \State \textbf{Operation} $ops[\;]$
            \EndStruct
            \Struct{NodeInfo}
            \State \textbf{Desc*} $desc$
            \State \textbf{int} $opid$
            \EndStruct
            \Struct{Node}
            \State \textbf{NodeInfo*} $info$
            \State \textbf{int} $key$
            \State ...
            \EndStruct
            \end{algorithmic}
    \end{multicols}
    \vspace{-0.15in}
\end{algorithm}

An example of lock-free transaction execution is illustrated in Figure~\ref{fig:lfttconflict}.
In the proposed approach, conflicts are detected at the granularity of a node. 
If two transactions access different nodes (i.e. the method calls in them commute~\cite{herlihy2008transactional}), they are allowed to proceed concurrently. 
In this case, shared-memory accesses are coordinated by the concurrency control protocol in the base data structure.
Otherwise, threads proceed in a cooperative manner.
At the beginning, Thread 1 committed a transaction $t1$ inserting keys 1 and 3 without any interference from the other threads. 
Thread 2 attempted to insert keys 4 and 2 in transaction $t2$, while Thread 3 was concurrently executing transaction $t3$ to delete keys 3 and 4.
Thread 3 was able to perform its first operation by updating the $info$ pointer on node 3 with a new \textsc{NodeInfo}.
In a transformed data structure, the logical status of a node depends on the interpretation of its transaction descriptor. 
An operation needs to change a node's logical status before performing any necessary low-level node manipulations. 
This is done by updating the node's \textsc{NodeInfo} pointer through the use of single-word CAS synchronization primitive. 
However, Thread 3 encounters a conflict when attempting to update Node 4 because Thread 2 has yet to finish its second operation.
To enforce serialization, operations must not modify an active node. 
In order to guarantee lock-free progress, the later transaction must help carry out the remaining operations in the other active transaction.
In this case, Thread 3 reads the transactionsaction descriptor stored in Node 4, and help Thread 2 finialize the insertion of Node 2 before updating the $info$ pointer on Node 4.

When applied to an existing base lock-free linked data structure, the lock-free transacitonal transformation process involves two steps: 1) identify and encapsulate the base data structure's methods for locating, inserting, and deleting nodes; and 2) Apply pre-defined code templates to construct the transformed operations.
The first step is necessary because the transformed data structure still relies on the base data structure's concurrency control to add, update, and remove linkage among nodes.
This is a refactoring process, as the functionality of the base implementations is not altered. 
Although implementation details such as argument types and return values may vary, one needs to extract the following three functions: \textsc{Do\_LocatePred}, \textsc{Do\_Insert}, and \textsc{Do\_Delete}.
A prefix \textsc{Do\_} is added to indicate these are the methods provided by the base data structures.
For brevity, detailed code listings are omitted. 
The general functionality specifications are express as follows. 
Given a key, \textsc{Do\_LocatePred} returns the target node (and any necessary variables for linking and unlinking a node, e.g., its predecessor).
\textsc{Do\_Insert} creates the necessary linkage to correctly place the new node in the data structure.
\textsc{Do\_Delete} removes any references to the node.

Algorithm~\ref{alg:listinsert} lists the template for the transformed \textsc{Insert} function.
The function resembles the base node insertion algorithm with a CAS-based while loop (line~\algref{alg:listinsert}{l:listinsertwhile}).
Compared with the \textsc{Insert} operation in the based data structure, the only code addition is a new code path for invoking \textsc{UpdateInfo} on line~\algref{alg:listinsert}{l:insupdateinfo}.
As mentioned above, the function \textsc{UpdateInfo} updates a nodes \textsc{NodeInfo} reference atomically.
The logic is straight forward: on line~\algref{alg:listinsert}{l:listinsertcontain} the code checks if the data structure already contains a node with the target key.
If so, it tries to update the node's logical status, otherwise it falls back to the base code path to insert a new node. 
Should any of the two code paths indicate a retry due to contention, it starts the traversal over again.

\begin{algorithm}[t]
    \caption{Template for Transformed Insert Function}
    \label{alg:listinsert}
    \begin{algorithmic}[1]
        \Function{Insert}{$\textbf{int}\;key,\;\textbf{Desc*}\;desc,\;\textbf{int}\;opid$}
        \State $\textbf{NodeInfo*}\; info \gets \text{\textbf{new NodeInfo}}$
        \State $info.desc \gets desc,\;info.opid \gets opid$
        \While{\TRUE} \label{l:listinsertwhile}
        \State $\textbf{Node*}\;curr \gets \Call{Do\_LocatePred}{key}$
        \If{\Call{IsNodePresent}{$curr,\;key$}} \label{l:listinsertcontain}
        \State $ret \gets \Call{UpdateInfo}{curr,\;info,\;false}$ \label{l:insupdateinfo}
        \Else
        \State $\textbf{Node*}\; n \gets \text{\textbf{new Node}}$
        \State $n.key \gets key,\;n.info \gets info$
        \State $ret \gets \Call{Do\_Insert}{n}$ \label{l:linknode}
        \EndIf
        \If {$ret = success$}
        \State \Return \TRUE
        \ElsIf {$ret = fail$}
        \State \Return \FALSE
        \EndIf
        \EndWhile
        \EndFunction
    \end{algorithmic}
\end{algorithm}
